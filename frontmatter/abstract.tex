Random numbers govern much;  they decide the fairness of a variety of lotteries and they are responsible for security in cryptographic schemes and modern communication. A randomness beacon is a service that emits random numbers at regular intervals. Everyone listening to this beacon will hear the same number, and this property enables transparency in applications that depend on randomness.% This is useful for auditing and verifying the use of randomness in many applications --- if the beacon is \enquote{honest} and sufficiently random, an entity drawing a lottery or designing cryptographic schemes will have no ability to bias or manipulate the process (at least not in the parts involving outcome of the randomness).

We conduct a survey of randomness beacons and their trustworthiness. We establish a general randomness beacon definition, and provide the, to our knowledge, first taxonomy of existing approaches to creating such beacons. We classify a total of 11 papers within 3 archetypes that cover trends within randomness beacons and similar work. The archetypes are also analyzed to further explore the main directions of approaches to public randomness.
