Random numbers govern much; for example, they decide the fairness of a variety of lotteries and they are responsible for security in cryptographic schemes and modern communication. A randomness beacon is a service that emits a random number at regular intervals. Everyone listening to this beacon will thus see the same number, and this property enables transparency in applications that depend on randomness.% This is useful for auditing and verifying the use of randomness in many applications --- if the beacon is \enquote{honest} and sufficiently random, an entity drawing a lottery or designing cryptographic schemes will have no ability to bias or manipulate the process (at least not in the parts involving outcome of the randomness).

This paper will be a survey of randomness beacons and their trustworthiness. The purpose is to establish a general randomness beacon definition, and to provide the, to our knowledge, first taxonomy of existing approaches to creating such beacons.

\mtjnote{TODO: \enquote{Results}}