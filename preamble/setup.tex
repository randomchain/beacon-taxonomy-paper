%!TEX root = ../main.tex

%%%%%%%%%%%%%%%%%%%%%%%%%%%%%%%%%%%%%%
% PAGE LAYOUT
%%%%%%%%%%%%%%%%%%%%%%%%%%%%%%%%%%%%%%

\usepackage{typearea}
\KOMAoptions{ % Basic page layout.
    %draft=true,
    paper=a4,
    abstract=true,
    twoside=true,
    fontsize=11pt,
    numbers=noenddot,
    BCOR=10mm % Margin binding correction
}

% Headers and footers
\usepackage[automark, headsepline=.3pt:head]{scrlayer-scrpage}
\pagestyle{scrheadings}
\ohead{{\MakeUppercase\leftmark}}
\ihead{\rightmark}
\setkomafont{pagehead}{\footnotesize\normalfont\sffamily}
\setkomafont{pagenumber}{\footnotesize\normalfont\sffamily}
\setkomafont{author}{\small\normalfont}
\setkomafont{date}{\small\normalfont}

%%%%%%%%%%%%%%%%%%%%%%%%%%%%%%%%%%%%%%
% FONTS, LANGUAGE AND TYPOGRAPHY
%%%%%%%%%%%%%%%%%%%%%%%%%%%%%%%%%%%%%%

\usepackage[english]{babel}
\usepackage{fontspec}
\usepackage[T1]{fontenc}

% Fonts
\usepackage[bitstream-charter]{mathdesign} % Serif font
\def\sfdefault{SourceSansPro-TLF} % Sans serif font
\usepackage{inconsolata} % Mono font

% Typographic tweaks
\usepackage[babel]{microtype} % Load after fonts. Yes, may be used with LuaLaTeX

% Space between line commands
\usepackage{setspace}

% Enabling various effects, such as highlightning
\usepackage{soulutf8}

%%%%%%%%%%%%%%%%%%%%%%%%%%%%%%%%%%%%%%
% ELEMENTS STYLES
%%%%%%%%%%%%%%%%%%%%%%%%%%%%%%%%%%%%%%

% Sections style
\setkomafont{section}{\normalfont\Large\fontseries{k}\sffamily\color{sectioning}}

% Subsections style
\setkomafont{subsection}{\normalfont\large\fontseries{k}\sffamily\color{subsectioning}}

% Subsubsections style
\setkomafont{subsubsection}{\normalfont\normalsize\fontseries{k}\sffamily\color{subsectioning}}

% Descriptions
\renewcommand{\descriptionlabel}[1]{\hspace{\labelsep}\color{descriptionlabeling}\sffamily{\textbf{#1}}}

% Captions
\KOMAoption{captions}{centeredbeside}
\setcapindent{1em}
\setkomafont{captionlabel}{\sffamily\color{sectioning}\bfseries}
\setkomafont{caption}{\sffamily}
\usepackage{subcaption} % Subfigures

%%%%%%%%%%%%%%%%%%%%%%%%%%%%%%%%%%%%%%
% FLOATING ENVIRONMENTS AND LISTS
%%%%%%%%%%%%%%%%%%%%%%%%%%%%%%%%%%%%%%

% Tables
\usepackage{booktabs}
\usepackage{multirow}
\renewcommand{\arraystretch}{1.1} % Stretch arrays (in tabulars)

\usepackage{enumerate}
\usepackage[inline]{enumitem}
\usepackage{outlines} % Apparently nice for nested lists

%%%%%%%%%%%%%%%%%%%%%%%%%%%%%%%%%%%%%%
% MATH
%%%%%%%%%%%%%%%%%%%%%%%%%%%%%%%%%%%%%%

\usepackage{amsmath}
\usepackage{amsthm}
%\usepackage[group-separator={,}]{siunitx}
\usepackage{fixmath} % Ensures math is typeset according to ISO standard

%%%%%%%%%%%%%%%%%%%%%%%%%%%%%%%%%%%%%%
% MISCELLANEOUS
%%%%%%%%%%%%%%%%%%%%%%%%%%%%%%%%%%%%%%

\usepackage{multicol}
    \renewcommand{\multicolsep}{\itemsep}
\usepackage[modulo]{lineno}
\usepackage{xspace}
\usepackage{datetime}
\usepackage{fvextra} %Keep this before csquote
\usepackage[autostyle]{csquotes}
\usepackage{glossaries}
\usepackage[super]{nth}

%%%%%%%%%%%%%%%%%%%%%%%%%%%%%%%%%%%%%%
% LATEX BACKGROUND STUFF
%%%%%%%%%%%%%%%%%%%%%%%%%%%%%%%%%%%%%%

\usepackage{xargs}
\usepackage{import} % Use relative imports

%Curfile stuff
%We need this to figure out the path if a command doesn't support import
\usepackage{currfile}
\makeatletter
\def\relativepath{\import@path}
\makeatother

%%%%%%%%%%%%%%%%%%%%%%%%%%%%%%%%%%%%%%
% GRAPHICS AND COLORS
%%%%%%%%%%%%%%%%%%%%%%%%%%%%%%%%%%%%%%

% Graphics
\usepackage{graphicx}
\graphicspath{{img/}}
\usepackage{rotating}

% Colors
\usepackage[table]{xcolor}

% Text colors
\definecolor{black}{gray}{0.0}
\definecolor{partcolor}{gray}{0.17}
\definecolor{chaptercolor}{gray}{0.17}
\definecolor{sectioning}{gray}{0.28}
\definecolor{subsectioning}{gray}{0.35}
\definecolor{descriptionlabeling}{gray}{0.35}

% Hyperlink colors.
\definecolor{pdflinkcolor}{RGB}{36,42,158}
\definecolor{urlcolor}{RGB}{133,52,52}

% Box colors.
\definecolor{clframe}{gray}{0.75}
\definecolor{clshade}{gray}{0.95}
\definecolor{clcodeshade}{gray}{0.93}
\definecolor{clnode}{gray}{0.85}

\definecolor{aaublue}{gray}{0}
\definecolor{bluekeywords}{gray}{0}
\definecolor{greencomments}{gray}{0.5}
\definecolor{redstrings}{gray}{0.3}
\definecolor{codebg}{HTML}{EFEFEF}
\definecolor{codefg}{HTML}{000000}
\definecolor{part}{HTML}{34495E}
\definecolor{numbers}{HTML}{34495E}
\definecolor{smartdiagram1}{HTML}{1ABC9C}
\definecolor{smartdiagram2}{HTML}{2ECC71}
\definecolor{smartdiagram3}{HTML}{3498db}
\definecolor{smartdiagram4}{HTML}{9b59b6}
\definecolor{smartdiagram5}{HTML}{E74C3C}
\definecolor{smartdiagram6}{HTML}{F1C40F}
\definecolor{smartdiagram7}{HTML}{E67E22}
\definecolor{diagramDark}{HTML}{19B5FE}
\definecolor{diagramLight}{HTML}{6BB9F0}

\definecolor{tableGoodLight}{HTML}{87D37C}
\definecolor{tableGoodDark}{HTML}{26A65B}
\definecolor{tableBadLight}{HTML}{EC644B}
\definecolor{tableBadDark}{HTML}{EF4836}

\definecolor{GoogleGreen}{HTML}{4CAF50}
\definecolor{GoogleRed}{HTML}{F44336}
\definecolor{GooglePurple}{HTML}{9C27B0}
\definecolor{GoogleDeepPurple}{HTML}{673AB7}
\definecolor{GoogleIndigo}{HTML}{3F51B5}
\definecolor{GoogleBlue}{HTML}{2196F3}
\definecolor{GoogleLightBlue}{HTML}{03A9F4}
\definecolor{GoogleCyan}{HTML}{00BCD4}
\definecolor{GoogleTeal}{HTML}{009688}
\definecolor{GoogleLightGreen}{HTML}{8BC34A}
\definecolor{GoogleLime}{HTML}{CDDC39}
\definecolor{GoogleYellow}{HTML}{FFEB3B}
\definecolor{GoogleAmber}{HTML}{FFC107}
\definecolor{GoogleOrange}{HTML}{FF9800}
\definecolor{GoogleDeepOrange}{HTML}{FF5722}
\definecolor{GoogleBrown}{HTML}{795548}
\definecolor{GoogleGrey}{HTML}{9E9E9E}
\definecolor{GoogleBlueGrey}{HTML}{607D8B}

\colorlet{Highlight}{GoogleBlue}


% Tikz and pgfplots
\usepackage{tikz}
\usepackage{pgfplots}
\usepgfplotslibrary{external}
%\usepgfplotslibrary{fillbetween}
\pgfplotsset{compat=1.14}
\usetikzlibrary{%
    calc, positioning, fit, arrows.meta
}
\tikzexternalize[prefix=figures/]
\tikzexternalenable{}

\tikzset{every picture/.style={/utils/exec={\sffamily}}}

\tikzstyle{node} = [thick, circle, draw, fill=clnode, text width=1.5em, text centered, minimum height=1.5em]
%\tikzstyle{block-textdepthfix} = [rectangle, draw, thick, fill=clcodeshade, text width=7.5em, text centered, minimum height=3em, text depth=.25ex]
\tikzstyle{block} = [rectangle, draw, thick, fill=clcodeshade, text width=7.5em, text centered, minimum height=3em]
\tikzstyle{arrow} = [draw, -{Stealth[length=2mm, width=2mm]}, thick]

%Two uml packages. I love it
%\usepackage{packages/tikz-uml}
%\usepackage{pgf-umlsd}

%%%%%%%%%%%%%%%%%%%%%%%%%%%%%%%%%%%%%%
% SOURCE CODE
%%%%%%%%%%%%%%%%%%%%%%%%%%%%%%%%%%%%%%

%Listing & Minted
\usepackage{shellesc}
\usepackage{listings}

%Minted makes nice looking code listings, but requires listings
\usepackage{minted}
\newminted[python]{python}{frame=leftline, framesep=2mm, linenos, fontsize=\footnotesize, baselinestretch=1.1, autogobble}
\usepackage{tcolorbox}

\usepackage{mdframed}
\mdfsetup{topline=false, rightline=false, leftline=false, bottomline=false}

%%%%%%%%%%%%%%%%%%%%%%%%%%%%%%%%%%%%%%
% FixMe
%%%%%%%%%%%%%%%%%%%%%%%%%%%%%%%%%%%%%%

\usepackage{fixme}
\fxsetup{%
    %status=draft,
    layout={footnote, index},
    innerlayout={inline, index},
    theme=colorsig,
    mode=multiuser,
    silent=true
}

\FXRegisterAuthor{msm}{msme}{Michno}
\FXRegisterAuthor{mtj}{mtje}{Michael}
\FXRegisterAuthor{srk}{srke}{Sebastian}

\input{preamble/fixme.tex} % Inputs status=draft??? Is this even necessary?

%%%%%%%%%%%%%%%%%%%%%%%%%%%%%%%%%%%%%%
% BIBLIOGRAPHY
%%%%%%%%%%%%%%%%%%%%%%%%%%%%%%%%%%%%%%

\usepackage[
    style=numeric,
    maxnames=1,
    backend=biber,
    defernumbers=true,
    hyperref=true,
    alldates=long,
    natbib=true
]{biblatex}
\bibliography{bibtex.bib}

\setcounter{biburlnumpenalty}{300} % Enable breaking URLs in bibliography after numbers
\setcounter{biburlucpenalty}{300} % Enable breaking URLs in bibliography after uppercase letters
\setcounter{biburllcpenalty}{300} % Enable breaking URLs in bibliography after lowercase letters


%%%%%%%%%%%%%%%%%%%%%%%%%%%%%%%%%%%%%%
% References
%%%%%%%%%%%%%%%%%%%%%%%%%%%%%%%%%%%%%%

\usepackage[english]{varioref} % \vref, if a thing is referred on another page.
\usepackage{hyperref} % Load late
\hypersetup{
    bookmarksnumbered,
    bookmarksdepth=1,
    pdfauthor={\GROUP},
    pdftitle={\TITLE},
    pdfpagelayout=TwoPageRight,
    pdfdisplaydoctitle,
    breaklinks,
    linktoc=all,
    pdfprintscaling=None,
    %plainpages=false,
    colorlinks,
    linkcolor=aaublue,
    citecolor=aaublue,
    urlcolor=aaublue,
    filecolor=aaublue
}

\usepackage[english, capitalise, noabbrev, nameinlink]{cleveref} % Load late
\crefname{appchap}{appendix}{appendices}
\Crefname{appchap}{Appendix}{Appendices}

% For refs to description itmes
\makeatletter
\def\namedlabel#1#2{\begingroup
    #2%
    \def\@currentlabel{#2}%
    \phantomsection\label{#1}\endgroup
}

%%%%%%%%%%%%%%%%%%%%%%%%%%%%%%%%%%%%%%
% EXTERNAL FILES
%%%%%%%%%%%%%%%%%%%%%%%%%%%%%%%%%%%%%%

% chktex-file 01 chktex-file 09 chktex-file 10 chktex-file 17

% Monospaced text that is beautiful
\newcommand*\justify{%
    \fontdimen2\font=0.4em% interword space
    \fontdimen3\font=0.2em% interword stretch
    \fontdimen4\font=0.1em% interword shrink
    \fontdimen7\font=0.1em% extra space
    \hyphenchar\font=`\-% allowing hyphenation
}
\newcommand{\mono}[1]{\texttt{\justify {#1}}}

% Code mono
\newcommand{\inlinecode}[1]{\sethlcolor{clcodeshade}\hl{\texttt{\justify{#1}}}} % Yes, this line is identical to the one in \mono, however the soul-package is fragile, and cannot see into another command.
\soulregister{\justify}{1}

\newmdenv[
nobreak=true,
suppressfirstparskip,
topline=false,
rightline=false,
bottomline=false,
linewidth=2.5pt,
linecolor=clframe,
backgroundcolor=clshade]{formal}

\makeatletter
\newtcbox{\code}{
    on line,
    fontupper=\small\ttfamily,
    boxrule=0.5pt,
    arc=2pt,
    coltext=codefg,
    colback=codebg,
    colframe=codebg,
    boxsep=0pt,
    shrink tight,
    extrude by=2pt
}
\makeatother

%Make \Chaptername return print the name of the chapter
\let\Sectionmark\sectionmark
\def\sectionmark#1{\def\Sectionname{#1}\Sectionmark{#1}}
\let\Subsectionmark\subsectionmark
\def\subsectionmark#1{\def\Subsectionname{#1}\Subsectionmark{#1}}
\let\Subsubsectionmark\subsubsectionmark
\def\subsubsectionmark#1{\def\Subsubsectionname{#1}\Subsubsectionmark{#1}}

\newlist{enumberate}{enumerate}{2}
\setlist[enumberate,1]{label=\itshape \arabic*\upshape)}
\setlist[enumberate,2]{label=\itshape \arabic{enumberatei}.\arabic*\upshape)}
\newlist{eletterate}{enumerate}{2}
\setlist[eletterate,1]{label=\itshape \alph*\upshape)}
\setlist[eletterate,2]{label=\itshape \alph{eletteratei}.\alph*\upshape)}
\newlist{eromanrate}{enumerate}{2}
\setlist[eromanrate,1]{label=\itshape \roman*\upshape)}
\setlist[eromanrate,2]{label=\itshape \roman{eromanratei}.\roman*\upshape)}

\newlist{enumberate*}{enumerate*}{1}
\setlist[enumberate*]{label=\itshape \arabic*\upshape)}
\newlist{eletterate*}{enumerate*}{1}
\setlist[eletterate*]{label=\itshape \alph*\upshape)}
\newlist{eromanrate*}{enumerate*}{1}
\setlist[eromanrate*]{label=\itshape \roman*\upshape)}

\SetLabelAlign{parright}{\parbox[t]{\labelwidth}{\raggedleft#1}}
% 1st param: font style of label, 2nd param: text of longest label
\newenvironment{dankscription}[2]{\begin{description}[labelindent=\parindent, labelwidth=\widthof{#1#2}, align=parright, font=\normalfont#1]}{\end{description}}

\newcommandx{\cnameref}[1]{%
\Cref{#1} \emph{\nameref{#1}}}

\newcommandx{\myref}[2][1=]{%
    \IfStrEq{#1}{name}{\Cref{#2} \emph{\nameref{#2}} \vpageref{#2}}{\Cref{#2} \vpageref{#2}}%
}

\newcommand{\tblgrpsep}{\noalign{\vspace{.75em}}}

\newenvironment{problemstatement}
{\begin{mdframed}[
        linewidth=1pt,
        linecolor=Highlight,
    leftline=true]}
{\end{mdframed}}

\newcommand{\stefan}[1]{\fxnote[nofootnote,inline,author=Stefan]{#1}}

\input{preamble/latexdiff.tex}

%%%%%%%%%%%%%%%%%%%%%%%%%%%%%%%%%%%%%%
% MISCELLANEOUS TO BE LOADED LAST
%%%%%%%%%%%%%%%%%%%%%%%%%%%%%%%%%%%%%%

\usepackage{scrhack} % Load very, very late. Renews commands.

