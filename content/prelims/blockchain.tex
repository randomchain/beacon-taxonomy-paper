\subsection{Blockchain}\label{subsec:blockchain}
The blockchain is an append-only public ledger that most notably serves as the backbone of the bitcoin cryptocurrency.
It is a peer-to-peer network within which users can perform transactions between each other.
Transactions are recorded into a chain of blocks, each pointing to the previous one, that make up the blockchain.
Users compete for the right to make the blocks, as the action is rewarded.
A consensus algorithm is applied to ensure a constant flow of correct, agreed upon blocks.
The correctness of each block is publicly verifiable by every user in the blockchain~\cite{nakamoto2008bitcoin}.

We consider the blockchain primarily for how it uses public verification to eliminate the need for trusted parties, is able to establish consensus, and how it provides a strong base for verifiable peer-to-peer interactions.
Some public blockchains also use proof-of-stake, which requires randomness, for their consensus protocols and coin minting.
These types of blockchains make an obvious beneficiary of publicly verifiable randomness beacons.
