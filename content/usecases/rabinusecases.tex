\subsection{Fair Contract Signing}\label{subsec:fair_contract_signing}
\stefan{maybe have this use case first as it was also the original one proposed?}
\citet{rabin1983transaction} motivated his invention of the beacon with an example.
If two parties, $A$ and $B$, want to negotiate a contract, they each want the other party to sign it.
However, if $A$ signs it and sends it to $B$, then $B$ will be in a position of strength, since $A$ has already committed to it, while $B$ is not.
$B$ can decide to sign it or abandon it.
Avoiding this situation is desired in a \emph{fair} contract signing protocol.

The usual way of overcoming this problem is using an intermediary trusted by both parties.
However, if no such intermediary can be found, Rabin proposes using a fair contract signing protocol that involves a beacon broadcasting random numbers.
The beacon output is used as the trusted intermediary to provide a known probabilistic guarantee that neither of the parties can cheat.
\mtjnote{Maybe we should be more specific?} \stefan{yes it is not clear what the idea is here}
