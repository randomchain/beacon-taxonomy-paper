\subsection{Cryptography}\label{subsec:usecase_cryptography}
Randomness and unpredictability are important parts of modern cryptography. 
%The strength of security protocols is often expressed as a number of bits an attacker needs to guess to execute a successful attack. It follows, that the harder it is to guess a bit, i.e.\ the higher the entropy of your data, the more secure your protocol is.
 However you must trust the source of randomness to not be biased or controlled by an adversary to trust the encryption. This makes a publicly verifable randomness beacon an excellent source of randomness for cryptography \stefan{any citations for that? also be careful: if the beacon is public and everyone uses the same random number, this can backfire: everyone will use the same crypto-keys!}. A straightforward use would be a protocol where participants need to agree on some random number - they can simply commit to the output of a beacon at some previously agreed timestamp. 
However, randomness is not usable for all kinds of cryptography --- e.g.\ private encryption keys should used from publicly available randomness \stefan{i dont understand}. 
%Randomness is also used as the foundation of cryptographic primitives like zk-SNARKs \srknote[inline]{TODO: Source}. The randomness at the foundation of this primitive allows proving information without revealing it.

\mtjnote[nofootnote,inline]{I think it would be a good idea to cite million-dollar-curve in this section because they write how it is usually done today, and how they would benefit from using a beacon! Using digits of pi or a hash of some obvious string is the "state of art" today, but a paper that million-dollar-curve cites shows that there is a large degree of freedom to manipulate this method.}