\subsection{Cryptography}\label{subsec:usecase_cryptography}
Many cryptographic algorithms contain hard-coded seeds.
There are essentially two ways an algorithm designer can provide these~\cite{baigneres2015trap}:
\begin{eromanate*}
    \item arbitrarily choosing a value, or
    \item randomly choosing a value that can be verified \emph{a posteriori}
\end{eromanate*}.
The first approach is used in many algorithms --- as an example, all hash functions in the \gls{sha} family use round constants, and in case of the Dual-EC-DRBG pseudo-random number generator, it has been shown that if the constant has been generated in a certain way, the knowledge of this alone can be used to predict all future outputs~\cite{dualec-paper}.
The second approach is utilized to limit arbitrary choices; assuming hash functions cannot be inverted, a hash function applied to some data can provide a seemingly random seed.
For example, one may use the hash of some digits of $\pi$. This is known as \enquote{nothing up my sleeve-numbers}.
\citet{bernstein2015manipulate} criticizes this approach: why exactly $\pi$, and not $e$, $\sin(1)$, or another seemingly innocent input --- and why \emph{this} hash function and not \emph{that} hash function?
\citet{backdoorsupmysleeve} illustrates with a working code example that there is a great amount of freedom in choosing these seemingly innocent constants.

Instead, \citet{baigneres2015trap} essentially proposes using a randomness beacon (though, in their use case, only utilized once).
The idea is that, if the algorithm designer announces: \enquote{at time $T$ in the future, I will use the beacon output to produce my seed for my algorithm}, the only way for the seed to be manipulated is if the beacon has been manipulated.
