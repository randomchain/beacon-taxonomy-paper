\subsection{Lotteries}\label{subsec:usecase_lotteries}
A use case, which intuitively invites for the usage of public verifiable randomness, is lotteries.
They can be generalized as, a scenario where a set of actors has some stake in the outcome of a random draw.
Usually the random draw is done by the entity organizing the lottery, and the randomness of the draw is produced in custody of said entity.
Moreover, more often than not, the generation of the randomness is done using proprietary machines, whether it be in software, ping pong balls flailing around in an acrylic box, or some other theatrical display.
This means that each participating actor must trust the organization behind the lottery, for them to perceive the random draw itself to be fair.

Lotteries does not only relate to random draws in a gambling oriented setting, but also scenarios such as drawing groups for sport events, or sortition\footnote{The selection of a subset, by means of drawing a random sample. E.g., this was used in ancient Athens to elect political officials.}.
Lotteries can also determine the faith of individuals, as it happened with the random draft of soldiers in USA during the Vietnam War; here it was quickly discovered that the shuffling of outcomes was inadequate, thereby resulting in a biased result\cite{starr1997nonrandom}.
Even if people trusted that the government had not botched the shuffling on purpose, the draft still was not fair, and the randomness could not be trusted.
