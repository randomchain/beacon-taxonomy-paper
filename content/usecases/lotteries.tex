\subsection{Lotteries}\label{subsec:usecase_lotteries}
Generalized, a lottery is a scenario where a set of actors has some stake in the outcome of a random draw. Examples of lotteries include traditional lotteries with a monetary prize; lotteries determining a prioritization order, like the annual NBA draft; and conscription lotteries determining the fate of individuals as happened in the US Military randomly picking soldiers (of which many would later be sent to the Vietnam War).

Usually the random draw is performed by the entity organizing the lottery, based on proprietary techniques, whether it be software, ping pong balls flailing around in an acrylic box, or another theatrical display. As such, each participating actor must trust the organization behind the lottery. The three examples in the previous paragraph have all been involved in some degree of suspicion; there are cases of manipulation of traditional lotteries~\cite{lotteryscandal-eddietipton, lotteryscandal-666}, the NBA draft is every year causing fans to claim it was manipulated \todo{Find examples}, and the military draft was, by statisticians, proved to be botched~\cite{starr1997nonrandom}.

This could be remedied by the lottery entity revealing (i) a (fair) function that maps the space of a randomness beacon output to the required lottery draw space and (ii) a time in the future at which the output from the randomness beacon will be used in said function to obtain the result. It is crucial that the function is revealed before the randomness beacon output is emitted. This removes the need for trust in the lottery entity (at least in the random draw process), and moves the trust issue to the randomness beacon.
