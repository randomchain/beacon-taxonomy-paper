\subsection{Blockchain --- Preventing Selfish Mining}\label{sub:blockchain_preventing_selfish_mining}
In proof-of-work blockchains, such as \gls{btc}, an adversary can perform an attack called \emph{selfish mining}.
This is a strategy, where a miner or colluding group of miners withhold blocks they mine, and publish these blocks at certain times to gain an advantage;
usually this means that the selfish miner can dictate the content of the blockchain.

\citet{heilman2014one} presents a way to make selfish mining significantly harder to do;
Previously, a selfish miner was required to possess 25~\% of the mining power to be able to carry out the attack. Heilman raises this to 35~\%.
This is done by using unforgeable timestamps in the blocks, which means that a selfish miner will be exposed for having withheld mined blocks.
Randomness beacons can thus be used to provide these unforgeable timestamps, given that a random number is published at sufficient intervals.
