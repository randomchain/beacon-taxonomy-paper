% Randomness beacons
\section{Definition of a Randomness Beacon}\label{sec:beacons}

As part of the goal of this paper, we define a trustless randomness beacon and the properties it must exhibit.
This definition needs to be sufficiently general to encompass a variety of implementations.
First, we look at the root of \emph{randomness beacons}.
The root is not trustless, and therefore we discuss the validity of the root, motivate it, and afterwards propose new properties that make a randomness beacon trustless.

% Rabin and the root of beacons
\citet{rabin1983transaction} introduced the term \emph{beacon}\footnote{We use the terms \emph{randomness beacon} and \emph{beacon} interchangeably.
These should not be confused with concepts like Bluetooth beacons --- however we do not distinguish between our randomness beacon and Rabin's beacon.} in 1983, and describes it as a \emph{trusted} third party that behaves \emph{honestly} and emits random numbers at a set interval, as a way to enable fair contract signing and secret disclosure.
He furthermore describes three properties for his early beacon:

\begin{description}
    \item[Unpredictable] No party should be able to have advance knowledge of the beacon's output.
    \item[Random] The beacon should use some physical method to produce random numbers.
    \item[Available] The beacon should be publicly available, and could have backups in case of failure or jamming.
\end{description}

This definition can be used for all our previously described use cases with the glaring exception that users need to trust it.
While we agree with the basics of these properties, we find the properties to be vague and poorly motivated.
Therefore, we specify and extend Rabin's description of a beacon, and provide a motivation for why these properties are needed.


