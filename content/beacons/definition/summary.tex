\subsection{Summary}

For sake of completeness, we reiterate our generic randomness beacon properties and combine them with the verifiability.
They then describe properties for a trustless randomness beacon:

\begin{description}
    \item[Unpredictable]
        Any user, $u$, should not be able to predict the output, $O_t$, with a larger probability than any other user, $u'$.
    \item[Unbiased]
        All possible outputs are equally likely --- i.e.\ the output is statistically close to samplings from a uniform distribution.
        Given that $p(x)$ is the probability of $x$, and $O = \left\{ {o_0, o_1, \ldots, o_n} \right\}$, the set of all possible outcomes, this property can be formalized as:
        $$ \forall o_i \in O \mid p(o_i) \approx \frac{1}{n}$$
    \item[Entropy Guarantee]
        There must be a guarantee on the level of output entropy.
        Use cases then know if there is enough entropy for their use.
    \item[Available]
        All users in a given setting should have access to the randomness beacon.
        This means both their ability to contribute with input, if applicable, and their access to the output.
    \item[Verifiable]
        Users should not need to trust \emph{any} output.
        A beacon needs to be designed such that it can prove its honesty instead of requiring blind trust.
        This can for example be accompanying the output with a proof that can be used to verify the correctness of the output, or giving the users to ability to contest the correctness of the output.
\end{description}

Each of these properties may or may not be fulfilled to some degree --- however an ideal trustworthy randomness beacon should have all of them.
% As an absolute minimum, a beacon must be unpredictable and have some level of min-entropy.
% It must also be available, but does not always need to be this; periodic availability could be sufficient for some use cases.
% Verifiability is only required if users do not want to trust the beacon, it could easily function without it.
%This definition is more flexible than the predecessors, as it does not require a randomness beacon to be publicly available, but rather introduces a \emph{setting} since beacons could also see use in non-public settings.
%However, it also poses a more strict definition of correctness, since it must be proved.
%It also generalizes the verifiability to simply concern the honesty of the beacon.
The definition does not specify any properties about interaction, security, or execution of the beacon and its protocol, only that it somehow has to prove its honesty.
These properties depend on the setting and requirements of a given beacon.

This work is concerned with the trust of a randomness beacon, and as such we do not discuss the function $f$ for transforming the input to the output other than in the light of trust --- \citet{bonneau2015bitcoin, dodis2004randomness} provide some insights on the technicalities of such functions.
