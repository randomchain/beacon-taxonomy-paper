\subsection{Attributes}
% Variation attributes of beacons
We define three categories, which can be used to describe attributes that vary between different types of beacons.
These sets of attributes cover \emph{entropy sourcing}, \emph{execution model}, and \emph{validation}.
They are based on the components of \cref{fig:abstract_beacon}, and will also be tied back to the previously defined properties of beacons.
In the later taxonomy of current randomness beacons approaches, these attributes will also be used for classification.

% Entropy sourcing
\subsubsection{Entropy Sourcing}
The input of a randomness beacon at a given time, $I_t$, is the source entropy of the outcome.
The following describe different ways for a randomness beacon to source the input.

\paragraph{Internal Input}
The beacon uses input such as background radiation or output from photon splitters, measured with some local device.
These sources potentially provide high entropy~\cite{nistbeacon}, but are impossible to reproduce and \emph{verify} for users.
This means that the claimed input from a supposedly random source, could be entirely fabricated by the beacon operator to bias or even have total control over the outcome.
Hence, complete trust in the honesty of the beacon operator is required.
Moreover, the \emph{unpredictability} can never be guaranteed, since the beacon operator potentially could have precomputed the outcomes from seemingly random inputs.

An upside to using internal input is that the \emph{availability} does not depend on some outside force, as seen from the perspective of the beacon operator.

\paragraph{User Input}
The protocol allows users to provide entropy.
This can potentially open up for direct manipulation, which in turn can enable adversaries to overcome the \emph{unpredictability}.

The \emph{guaranteed entropy} of the inputs is dependent of the computation, i.e.\ aggregation of inputs and extraction of randomness.
Using \enquote{exclusive or} to combine the inputs guarantees that the lower bound of entropy is set by the input with the highest entropy~\cite{lenstra2015random};
this means that a user providing good entropy, can be sure that the outcome will have at least the same level of entropy.

Usually, the beacon must publish a commitment to a set of user inputs before beginning the randomness computation,  thereby making users able to reason about the inputs used.
Given knowledge of the algorithm used in the computation and the output, users will be able to \emph{verify} that a committed set of inputs has been used in the computation of the outcome.

If user input is the only source of entropy, lack of users prevent the beacon from operating, denying \emph{availability}.
There can also be too many inputs, resulting in input sourcing cutoff ---
the input of some users will then not be used to compute the outcome, thereby undermining that set of users' ability to trust the given outcome.
This is because a given user must be able to identify the presence of a trusted input in the committed set of inputs, to trust the beacon outcome.

\paragraph{Publicly Available Input}
The beacon uses some publicly available data for input.
Publicly available input means that it must be accessible to all users, and unpredictable until published.
Hence, the \emph{availability} of the beacon is dictated by the availability of the source of public input.
Moreover, the \emph{guaranteed entropy} is also determined by the input, which should be scrutinized to guarantee the lower bound of entropy.
These sources can for example be the bitcoin blockchain~\cite{bonneau2015bitcoin, bentov2016bitcoin}, financial data~\cite{clark2010use}, or national lottery results~\cite{baigneres2015trap}.

Anyone who wishes to manipulate the beacon's output must do so through the source.
For this type of source it is important to analyze the integrity of the source to understand the integrity of the randomness beacon itself.

It is important to note that public input does not violate the \emph{unpredictability} property of the beacon, since the time advantage an adversary has over other users is negligible regardless of computing resources.
Furthermore, any user of the beacon will be able to verify that the input has been used to generate the outcome.

\subsubsection{Execution Model}
Once the input entropy is obtained, some computation, $f(I_t)$, is performed to extract near uniform randomness for the outcome.
We identify the following ways of executing such a computation in a randomness beacon protocol:

\paragraph{Self-Announced Entity}
The protocol is computed by a central entity who provides a service in form of a beacon.
This type of execution requires \emph{verifiability} or complete trust in the central entity.

The entity, while self-announced, can be driven by a community, authority, or single person, but is always controlled by that entity.
\emph{Availability} can be a concern since this execution model imposes a single point of failure, if not for computation then for trust.

Users of a beacon operated by a self-announced entity, should consider the \emph{unpredictability} of the beacon, since the self-announced entity inherently will know the outcome before publishing.
This can be mitigated by the operator by committing to a given input before performing the computation.
However, this only guarantees that no input manipulation by the operator has happened between the commitment and publishing the outcome.

\paragraph{Elected Operator}
A user is elected through some leader election protocol to be the operator which performs the computation.
That user then executes the beacon protocol as a public good.
This type of execution participation requires that the operator has an incentive to carry out the beacon protocol, and may hurt \emph{availability} if the operator is unable or unwilling to complete.

The same question of \emph{unpredictability} arises as with the self-announced entity, and must be addressed by the beacon protocol to ensure trustworthiness.

\paragraph{Distributed Execution}
The computation of the protocol is done in a distributed manner between a set of parties e.g.\ using \gls{mpc} and/or smart contracts.
An example could be collecting inputs from each participant and performing \enquote{exclusive OR} operations between inputs.
This model potentially does not require trust in any other participant depending on the execution scheme.
However, scalability to large settings may prove difficult, and participants with byzantine behavior must be accounted for, which can hurt execution time and \emph{availability}.

\paragraph{Self-Service Execution}
Each user performs the same execution to obtain a random value.
The beacon is provided as a self-service function.
This model does not require trust in the executing participants, only in the input entropy, and manipulation can become prevalent with potential predictability if using user input.
Computational resources are only consumed by participants with a stake in the beacon output.

Uniquely, the \emph{availability} of this execution model is solely determined by the input source.
Also, this model requires users to potentially verify other's outcome, if disputes happen.

\subsubsection{Validation}
The output of the computation in the randomness beacon need some form of validation in order to be trustworthy.
We identify the following ways of validating a beacon:

\paragraph{Verifiable}
Any user can verify the outcome to have been correctly computed from the provided entropy input.
The verification process might include correlating the outcome with an input set the beacon has committed to.
This is directly related to the \emph{verifiability} in the earlier beacon definition.

\paragraph{Contestable}
Users can contest the output of the beacon if they believe it to be wrong.
The beacon operator must then prove the correctness or incur some penalty, while a user that successfully contests the beacon is rewarded.
This can potentially degrade the beacon \emph{availability}, depending on how the contesting process is handled and executed.
