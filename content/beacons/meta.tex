% Randomness beacons
\section{Randomness Beacons}\label{sec:beacons}
%   What is a beacon?

In the context of this paper \mtjnote{What does this mean?}, we seek to define a randomness beacon.
The term was originally introduced by \citet{rabin1983transaction}, who describes it as a trusted third party that emits random numbers at a set interval as a way to enable fair contract signing and secret disclosure.
He assumes that beacon is provided as a service by a trusted third party and behaves \enquote{honestly}. He furthermore argues for a number of criteria for the beacon:

\begin{description}
    \item[Unpredictability] The output of the beacon should be unpredictable before it is published.
    \item[Random] The beacon should use some physical method to produce random numbers.
    \item[Availability] The beacon should be publicly available, and could have backups in case of failure or jamming.
\end{description}

As stated, this definition requires trust in a third party, with no way to ensure the honesty of that party and closely matches the \gls{nist} randomness beacon.
Since we do not want to trust blindly in a third party, we will look for an alternative definition.

The definition is expanded upon by \citet{bonneau2015bitcoin}, who describe a decentralized beacon based on the Bitcoin blockchain.
It is a function that returns an $m$-bit near-uniformly distributed random value $r$ at each time interval $t$.
No assumptions are made on the source of randomness beyond a lower bound on min-entropy.
The $m$-bit value is computed from a sample of the source at time $t$, $D_t$, using an extractor function.

They describe a beacon as a function applied regularly to some input with a known min-entropy. They capture an element necessary for a trustless randomness beacon, namely verifiability --- the ability to verify certain properties of the output.
They also relax the requirement on randomness, as the output must simply be statistically close to uniform random string.

Their beacon definition is however not broad enough to allow for some solutions. Specifically, they require the input to be unknown before time $t$, and they require that any user can compute the output after time $t$ --- two things that are not necessarily required in some solutions.

We therefore synthesize our own definition, which will be used forward in this paper:

A \emph{randomness} beacon is a service \mtjnote{or function?} that publishes random data with a known min-entropy. The random data is computed from some input, at a regular known interval.
Formally, let $B: f(I_t) \rightarrow R$, where $B$ is a beacon, $I_t$ is the input at time $t$, $f$ is an extractor function, and $R$ is the random outcome.
For $B$ to be a beacon it is then run at every $t_\Delta$ time interval, with new input $I_t$.
%The function $f$ can be an extractor function, as explained in \Cref{sub:prelims_randomness}.
Beacon $B$ must exhibit the following properties:

\begin{description}
    \item[Unpredictability] Any user is unable to predict any information about the output of the beacon prior to time $t$. This indirectly also makes the beacon unmanipulatable, as manipulation would lead to predictability for some user(s).
    \item[Random] The output is statistically close to samplings from a uniform distribution.
    \item[Availability] All users in a given setting has access to the randomness beacon.
    This means both their ability to contribute with input, if applicable, and their access to the output.
    \item[``Proof of Honesty/Correctness''] Users should not trust \emph{any} output. A beacon needs to be designed such that it can prove its honesty instead of requiring blind trust. This proof can for example be a clever design enforcing unmanipulatable output, or accompanying the output with a proof that can be used to verify the correctness of the output.
\end{description}

This definition is more flexible than the predecessors, as it does not require a \emph{randomness beacon} to be publicly available, but rather introduces the term setting as beacons could also see use in non-public settings.
%It also generalizes the verifiability to simply concern the honesty of the beacon.

The definition does not specify any properties about interaction, security, or execution of the beacon and its protocol --- only that it somehow has to prove its honesty.
These properties rely on the setting and requirements of a given beacon.
