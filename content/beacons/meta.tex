% Randomness beacons
\section{Randomness Beacons}\label{sec:beacons}
%   What is a beacon?

As part of the goal of this paper, we model attributes a trustworthy randomness beacon must exhibit.
These attributes need to be sufficiently general to encompass a variety of implementations.
However before doing so, we will draw inspiration from earlier definitions.

The term \emph{beacon} was originally introduced by \citet{rabin1983transaction}, who describes it as a trusted third party that emits random numbers at a set interval as a way to enable fair contract signing and secret disclosure.
He assumes that beacon is provided as a service by a trusted third party and behaves \enquote{honestly}.
He furthermore argues for a number of criteria for the beacon:

\begin{description}
    \item[Unpredictability] The output of the beacon should be unpredictable before it is published.
    \item[Random] The beacon should use some physical method to produce random numbers.
    \item[Availability] The beacon should be publicly available, and could have backups in case of failure or jamming.
\end{description}

As stated, this definition requires trust in a third party, with no way to ensure the honesty of that party and closely matches the \gls{nist} randomness beacon.
Since we do not want to trust blindly in a third party, we will look for an alternative definition.

\citet{bonneau2015bitcoin} extends this description to describe a decentralized beacon based on the bitcoin blockchain.
It is a function that returns an $m$-bit near-uniformly distributed random value $r$ at each time interval $t$.
An assumption on the min-entropy of the input is made.
The $m$-bit value is computed from a sample of the source at time $t$, $D_t$, using an extractor function.

They describe a beacon as a function applied regularly to some input with a known min-entropy.
They capture an element necessary for a trustless randomness beacon, namely verifiability --- the ability to verify certain properties of the output.
They also differ in the requirement on randomness, as the output must simply be statistically close to uniform random string.

Their beacon definition is, however, tailored to describe their solution, and is as such not broad enough to allow for other solutions.
Specifically, they require the input to be unknown before time $t$, and they require that any user can compute the output after time $t$ --- two things that are not necessarily required in some solutions.

We therefore synthesize our own definition, which will be used forward in this paper:

A trustworthy randomness beacon is a service that publishes random data at a known interval.
Formally, let $B: f(I_t) \rightarrow R$, where $B$ is a beacon, $I_t$ is the input at time $t$, $f$ is a suitable function for generating randomness (such as an extractor function), and $R$ is the random outcome.
For $B$ to be a beacon, it is then run at a known, regular interval, $\delta$, such that $t+n\delta$ for any $n \in \mathbb{N}$ are valid times.
The quality of $R$ depends on the quality of $I$.
It is, however, difficult to quantify these qualities without making assumptions on the function $f$ or assumptions about the applications using the output of the beacon --- we will therefore not require a quantity, but rather allow for reasoning about the quality of a given beacon.


To describe a trustworthy randomness beacon, we present the following attributes.
Each of these attributes may or may not be fulfilled to some degree --- however an ideal trustworthy randomness beacon should have all of these attributes.

\begin{description}
    \item[Unpredictable]
        Any user should not be able to predict any information about the output of the beacon prior to time $t$ --- or more precisely, a user should not be able to predict the output with a larger probability than sampling a uniform distribution.
        This indirectly also makes the beacon unmanipulatable, as manipulation would lead to some degree of predictability for some user(s).
    \item[Min-Entropy Reasoning]
        The output should be reasoned about to make sure it contains enough min-entropy.
        This is a way of guaranteeing some level of randomness.
    \item[Availability]
        All users in a given setting should have access to the randomness beacon.
        This means both their ability to contribute with input, if applicable, and their access to the output.
    \item[\enquote{Proof of Honesty}]
        Users should not trust \emph{any} output.
        A beacon needs to be designed such that it can prove its honesty instead of requiring blind trust.
        This proof can for example be accompanying the output with a proof that can be used to verify the correctness of the output, or giving the users to ability to contest the correctness of the output.
\end{description}

This definition is more flexible than the predecessors, as it does not require a randomness beacon to be publicly available, but rather introduces a \emph{setting} since beacons could also see use in non-public settings.
However, it also poses a more strict definition of correctness, since it must be proved.
%It also generalizes the verifiability to simply concern the honesty of the beacon.

The definition does not specify any properties about interaction, security, or execution of the beacon and its protocol --- only that it somehow has to prove its honesty.
These properties rely on the setting and requirements of a given beacon.
