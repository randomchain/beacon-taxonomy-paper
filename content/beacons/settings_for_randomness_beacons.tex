\subsection{Settings for Randomness Beacons}
%   Settings for beacons:
Since randomness beacons are a versatile and broad concept, they can be deployed in vastly different settings.
These settings all alter the way the beacon operates and how users interact with it.
Furthermore, it affects which shape the beacon implementation takes, i.e.\ different technologies better suit varying settings.

There are two main distinctions when dealing with settings of a randomness beacon --- \emph{permissioned} and \emph{permissionless}. These settings are defined by the set of users, $U$, allowed to use and interact with the beacon. If, within the universe of users $\forall\notin U$ the beacon is permissioned, and requires some kind of access control.
Within these categories are then different views on trust, scalability, intent, and incentives.

\subsubsection{Permissioned Setting}
In a permissioned setting it is safe to assume no participant will want to risk being caught red-handed acting maliciously, as it would revoke their permitted access to the beacon. This makes it less likely that they will directly try to cheat a beacon protocol.
However, we can suspect that an adversary will try to manipulate the outcome of a randomness beacon if possible --- all users are anticipated to seek whatever benefits themselves.
This means that all detectable manipulation and malicious activities will be avoided by the participants of the private network.

We also assume that the continuous operation of a randomness beacon is a priority for the permissioned network, which means that all participants have an incentive to ensure timely execution.
Unless a participant can benefit from the beacon not outputting, in which case they might simulate e.g.\ network issues to disrupt the beacon.

Scalability might be an issue in larger permissioned networks, however, generally permissioned networks are small enough that scaling is not a great concern.

\subsubsection{Permissionless Setting}
No guarantees can be made when it comes to permissionless networks, such as the internet.
We can assume that all participants will try to manipulate the beacon outcome to their advantage, no matter the cost.
Whether they have the resources available to do so, is besides the point.
Likewise, any individual user might also in some cases suspect that all other users are colluding against them.

In the permissionless setting, all participants may not have a stake in the randomness beacon operating constantly.
Therefore, it is important to have established incentives for both users to use the beacon, but also for the execution of the beacon protocol if it relies on users.

Because permissionless settings are almost infinitely large, i.e.\ any user should be able to join, scalability is a key concern --- availability requires scalability.
