\section{Conclusion}\label{sec:conclusion}
Throughout this paper, we have examined the topic of public randomness with a focus on randomness beacons and trust. We have explored the beacon concept as well as a variety of previous work that either was or resembles a randomness beacon in structure. 
From this work we identify the following contributions:

\begin{description}
    \item [Beacon Definition] We provided a definition of a trustless randomness beacon, which expands on previous beacon definitions. It also serves as a more generic definition of the concept, that is not tied to a specific implementation. 
    \item [Beacon Comparison Parameters] We established a variety of comparison parameters in the form of variations for randomness beacons and similar approaches. The parameters and their effect on a beacon was also explored. 
    \item [Classification of Existing Approaches] Based on our parameters, we compared 11 existing approaches to public randomness. They were distributed within three archetypes of beacons. Each archetype had it's strengths and weaknesses analyzed to better explore the effect of certain beacon properties. 
\end{description}

The main contribution of the paper is the comparison and classification of existing papers. By analyzing and organizing them we have given a structured overview of the field and provided a framework that can be used to classify future work according to our parameters. 
The beacon definition is another contribution that can be used in future work as a generic definition of the term, which can clearly communicate the core of a randomness beacon. 