\section{Conclusion}\label{sec:conclusion}
In this paper, we have examined the topic of public randomness with a focus on randomness beacons and trust.
We have explored the beacon concept as well as a variety of previous work that either was or resembles a randomness beacon in structure.
From this work we identify the following contributions:

\begin{description}
    \item [Beacon Definition] We provided a definition of a trustless randomness beacon, which expands on previous beacon definitions.
It also serves as a more generic definition of the concept, that is not tied to a specific implementation.
    \item [Beacon Comparison Parameters] We established a variety of comparison parameters in the form of variations for randomness beacons and similar approaches.
The parameters and their effect on a beacon was also explored.
    \item [Classification of Existing Approaches] Based on our parameters, we compared 11 existing approaches to randomness beacons.
They were distributed within three archetypes of beacons.
Each archetype had strengths and weaknesses analyzed to explore the effect of certain beacon properties.
\end{description}

Based on an analysis and organization we gave a structured overview of the field and provided a framework that can be used to classify future beacon approaches according to our parameters.
The beacon definition can be used in future scientific work as a generic definition of the term, which can clearly communicate the core of a randomness beacon, including variations in different approaches.

Our work reveals a number of trade-offs regarding randomness beacons.
It is clear that beacons of the \emph{Autocratic Collector} archetype cannot be trusted for public randomness, and the \emph{Specialized \gls{mpc}} archetype does not scale well.
This makes it difficult to use for public settings if all users must participate to trust the result.
The combination of these makes the \emph{Transparent Authority} archetype the most suited for a trustless public randomness beacon.

This type of beacon typically relies on a third party source of input to generate randomness from.
Throughout the literature, the bitcoin blockchain has been a popular source of input due to the randomness inherent in the proof-of-work consensus protocol~\cite{bonneau2015bitcoin}.
Despite this, the blockchain has previously been shown to be malleable, and a variety of attacks on such a beacon exist~\cite{bonneau2015bitcoin, bentov2016bitcoin, pierrot2016malleability}.
However it has been shown, that the malleability can be mitigated significantly~\cite{bunz2017proofsof}, resulting in a probabilistically trustworthy beacon.

An alternative approach for a \emph{Transparent Authority} could be to use user input as \citealias{lenstra2015random} does.
In such a beacon, users are guaranteed that the results is not biased against them as long as their input is included.
While this does incur the risk of getting only bad or no input, they also suggest using a third party source to ensure a minimum level of entropy in the output.

A combination of user input and blockchain could serve as a powerful beacon; providing the scalability, availability and trustworthiness of a \emph{Transparent Authority}, while allowing users to provide input to achieve guaranteed trust, and using the blockchain to provide make manipulation improbable and guarantee an entropy level.
