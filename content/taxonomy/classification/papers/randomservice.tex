\item[A Public Randomness Service]
\citet{fischer2011publicrandomnessservice} describe how the world has a need for a public source of randomness. They outline a protocol for a public trusted randomness server on the internet, that serves randomness to users in the style of an autocratic collector beacon. They generate random numbers by performing an exclusive or operation on numbers from a pool of trusted servers. They discuss how to control membership of the pool, and how to prevent servers from committing last-draw attacks, but do not implement their protocol. Ultimately they do not require their protocol to be verifiable, and neither do they specify what servers would qualify as trusted, which makes the protocol hard to trust. 