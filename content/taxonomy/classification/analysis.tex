\subsection{Analysis}
From our classification in \Cref{tab:paper_overview} a number of trends emerge in beacons and similar approaches to public randomness. The NIST beacon \cite{nistbeacon} is the one of the most well-known beacons available today, and publishes random strings every 60 seconds. It is a self-announced entity with private input, and no option for validation. These properties make it excellent for producing high-quality randomness and trivial to scale, but the lack of verification means the beacon requires blind trust in the NIST. Similarly, \citet{fischer2011publicrandomnessservice} (A Public Randomness Service) shares the same properties, although the paper is purely theoretical and does not implement a beacon. Interestingly, it is partially written by scientists from the \gls{nist}, which may explain its overlap with their beacon. These are clear examples of the autocratic archetype, that operate highly efficiently, but lack verifiability. 
Another trend that is clear is using publicly available data and some form of verification to provide trustable randomness. Examples of this trend are \citet{clark2010use} (Financial Data) , \citet{baigneres2015trap} (Million Dollar Curve) , \citet{bonneau2015bitcoin} (On bitcoin) , and \citet{bentov2016bitcoin} ($\pi_\text{BEACON}$). Among these we find both self-announced and self-service beacons, and all save one are also verifiable. Overall they closely match the transparent authority archetype, which has good scalability, but whose availability and throughput is tied to it's source of input. The archetype also leverages verifiability to remove the need to trust the authority. The self-service examples\cite{bonneau2015bitcoin} \cite{bentov2016bitcoin}  remove the central authority to simply provide a function for users to calculate the randomness. This requires users to calculate the randomness themselves if they want to agree on it, but otherwise functions similarly to the others. Another variation on this is to use distributed execution with an elected leader, as \citet{bunz2017proofsof} (Proof-of-delay) show. They use an open competition model, where members of the network compete to compute and publish the beacon result the fastest for a reward. In addition, their output is contestable, and they establish incentive structures that support correct behaviour, i.e. publishing correct randomness, and challenging incorrect randomness. Their approach is still close to the transparent authority, only with constant competition to be the authority, and the possibility to despose malicious authorities. 
These examples are all dependendant on their input source for the quality of their randomness. Many of them use block-hashes from the bitcoin blockchain for this \cite{bonneau2015bitcoin, bentov2016bitcoin, bunz2017proofsof}, as the proof-of-work puzzles employed there have been shown to contain large amounts of entropy\cite{bonneau2015bitcoin}.  %Alternatives ? + proof-of-work discussion + blockchain malleability
Another trend is using user input - examples of this are \citet{syta2017scalable} (Randhound), \citet{cascudo2017scrape} (SCRAPE), \citet{randao} (RANDAO) and \citet{lenstra2015random} (Random Zoo). These examples also vary quite a bit inbetween themselves, as it has \gls{mpc} models \cite{syta2017scalable, cascudo2017scrape}, as smart contract \cite{randao} and an example that is closest to a transparent authority \cite{lenstra2015random}. The \gls{mpc} examples have protocols to generate randomness from input within a group of users. Their protocols are based on Shamir Secret Sharing Schemes that allow them to securely share encrypted input to generate randomness. However, such protocols are usually hard to scale, as their complexity increases with the amount of users. 
In addition, protocols based on user input can be hard to trust for users that have not contributed to the randomness. Even if commits are published to a set of user inputs, there is no guarantee that all of the input was not supplied by colluding adversaries. In addition, the protocols will fail without any users to generate randomness, and SCRAPE \cite{cascudo2017scrape} even requires an honest majority to succeed. This makes this type of beacon difficult to deploy in a public setting where any number of users should be able to use and trust them. However, they can perform excellently in smaller settings where it is easier for all users to consistently contribute. 
The RANDAO\cite{randao} is a smart contract on the ethereum blockchain that provides randomness to other smart contracts at a small fee. Users can contribute input to the contract through transactions, but are required to deposit small amounts of currency as a bond. The bond is lost if users diverge from the protocol, but any profits from the contract providing randomness is paid to contributing users. Like the previous examples, the protocol can still fail to provide randomness if no users contribute input. As with any ethereum smart contract, the correctness is guaranteed by every full node running the smart contract. 
Finally, \citet{lenstra2015random} use user input to create randomness by collecting it from some public channel, commiting to it, and running a slow extractor function on it. Their protocol is executed by a self-announced central authority, which makes it easy to scale like the transparent authorities, as the output simply needs to be posted publicly. In addition to user input, they suggest using some publicly available input - this also allows them guarantee output as long as that source is available, thus avoiding a common pitfall of user input; missing users. 
Another danger of user input is last-draw attacks, which they resist by using a slow extractor function - an adversary can not quickly extract randomness from all but the last input to determine which input would bias the randomness towards his benefit. Hence this example is a strong alternative to the trnasparent authorities with the added benefit that contributing users can be sure the randomness is not biased against them. 
%Den her sektion kunne godt være for fluffy / handwavy, selvom den refererer tilbage til tidligere begreber - kan omskrives senere. 