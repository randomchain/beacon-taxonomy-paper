\subsection{Comparison Parameters}
\label{sub:comparison_parameters}
To be able to compare different approaches to creating a randomness beacon, we establish the following parameters.

Firstly, we look at the \emph{attacker model} ---
What can an attacker do to harm the beacon, and what kind of resources does it cost.
The focus will be on two seperate attack vectors: \emph{availability}, and \emph{integrity}.
Availability concern attacks as \gls{dos}, i.e.~how can an adversary deny other users from accessing or verifying the beacon.
Integrity revolves around the attacker's ability to tarnish the outcome of the randomness beacon.
This is also closely related to the \emph{trust assumptions}, which examines what it takes for a given user to trust the input and output of the beacon.

Lastly, we look at the \emph{participation model} which describes how users interact with the beacon, both in input collection, output publication and general beacon verification.
