\section{Comparison Parameters}\label{sub:comparison_parameters}
\stefan{what is this section about?! the title does not reveal that => not a good title}

\stefan{here you should say somehting like: motivated by our use cases discussed before and the intuitions we gained, let us make things more formal in this section}

To be able to compare different approaches to creating a randomness beacon, we establish the following parameters.
\stefan{first formally define the requirements before you discuss how to achieve them}
%
%\mtjnote[nofootnote, inline]{More text describing this.}
%
%Firstly, we look at the \emph{attacker model} ---
%what can an attacker do to harm the beacon, and what kind of resources does it cost?
%The focus will be on two separate attack vectors, \emph{availability} and \emph{integrity}.
%Availability concerns attacks such as \gls{dos}, i.e.\ how an adversary can deny other users from accessing or verifying the beacon.
%Integrity revolves around the attacker's ability to tarnish the outcome of the randomness beacon.
%This is also closely related to the \emph{trust assumptions}, which examine what it takes for a given user to trust the input and output of the beacon.
%
%Lastly, we look at the \emph{participation model} which describes how users interact with the beacon regarding input collection, output publication and general beacon verification.
