\subsection{Attack Goals And Models}
\label{sub:attack_goals_and_models}
When looking at the goals an attacker trying to corrupt a randomness beacons from a high-level perspective, there exist two prevalent concepts: \emph{availability} and \emph{integrity}.
Both of these are also properties that a randomness beacon should exhibit, and each term embodies multiple actual attacks and requirements for fulfillment.

\begin{description}
    \item[Availability:]
        Attacks on the availability are essentially preventing a set of users from being able to gain access to a given randomness beacon.
        This can be both their ability to contribute with input, or discover the output.
        Moreover, in scenarios where the beacon is driven by a single entity, attacking the availability could also mean preventing that entity from collecting input or publishing the outcome.
        Thereby effectively denying service for all potential users.

        Examples of concrete attacks on the availability of a randomness beacon are:
        \acrfull{dos} where an attacker overwhelms the beacon with e.g.\ input, thereby making any other user unable to supply entropy as input, a \gls{dos} attack could also be on the availability of the source of input itself; eclipse attacks where an attacker eclipses a part of the network, thus only making the beacon unavailable to some select users.
    \item[Integrity:]
        The integrity of a randomness beacon, determines whether or not, a given user should trust the outcome, and to what degree.
        Attacking the integrity is therefore, any attack which compromises the outcome of a randomness beacon.
        This can both mean manipulating the input/entropy to the beacon, but also compromising the entire computation and outcome;
        as beacons can be seen as entities and not a particular actor, one might image a scenario where an attacker imposes themselves as the beacon.

        Attacks on the integrity could be: successfully supplying valid input, which is able to bias the outcome, e.g.\ last draw attacks, where an adversary supplies the last input and therefore potentially is able to choose some input beneficial to them; if the operator of the beacon is malicious, they might be able to generate a seemingly fair and random outcome by corrupting the computation of the beacon.
\end{description}


