\subsection{Participation Model}
\label{sub:participation_model}
These are the categories of participation models we use for our taxonomy: 

\begin{description}
    \item [User Input] The protocol allows users to provide entropy for the entropy. This can open up for direct manipulation of the result by last-draw attacks and the like. If this is the only source of entropy, lack of inputs denies availability of the randomness.
    \item [External Input] Using external input such as background radiation and output from photon splitters. These sources are, however, difficult to reproduce and verify.
    \item [Publicly Available External Input] The protocol uses some publicly available external source of entropy to produce randomness. These sources can be the blockchain, financial data,or national lottery results. Anyone wishing to influence it must influence the external source.
    \item [Distributed Computation]
    The computation of the protocol is done in a distributed manner between a set of parties. 
    \item [Central Operator] The protocol is computed by some central operator. For obvious reasons, this approach can not be trusted unless the computation is publicly verifiable. 
\end{description}

\mtjnote{Maybe we should split this into interaction participation and execution participation}