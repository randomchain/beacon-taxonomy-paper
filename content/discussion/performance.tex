\subsection{Beacon Performance}
One aspect of beacons we have not given much attention in our analysis is their performance.
In practical use of a beacon, the performance of the beacon is also an important parameter to consider.
When considering performance, the output frequency of the beacon is the primary metric to consider.
Depending on the setting, having fresh randomness available frequently could be important for e.g.\ fair contract signing\cite{rabin1983transaction}.
We briefly mention it during the analysis, but the frequency of a beacon can be correlated to our main archetypes of beacons.
The \emph{autocratic collector} practically dictates their frequency themselves.
They are limited only by the speed at which they can collect data.
To exemplify this, the \gls{nist} beacon output strings every 60 seconds~\cite{nistbeacon}.

The \emph{transparent authority} is also limited by their input source, but in their case the limitation is much greater.
They require the source to be published by a third party before they can start their computation, which puts the frequency out of their control.
This is best exemplified with the examples that use the bitcoin blockchain as source~\cite{bonneau2015bitcoin, bentov2016bitcoin, bunz2017proofsof}.
The bitcoin blockchain is designed to produce a new block at a set rate~\cite{nakamoto2008bitcoin}, which currently aims for a new block every 10 minutes.
This frequency may be too low for some usecases, in which case another type of beacon would be preferable.

By contrast, the frequency of \emph{Specialized MPC} beacons is completely dependant on the participants.
As the randomness is produced from their inputs, the frequency can be as low or high as the participants decide to set it.
