\subsection{Beacon Performance}
One aspect of beacons we have not analyzed much is their performance.
In practical use of a beacon, performance is an important parameter to consider.
When considering performance, the output frequency of the beacon is the primary metric we consider.
As briefly mentioned in the analysis the frequency of a beacon can be correlated to the archetypes.
The \emph{autocratic collector} practically decide their frequency themselves.
They are limited only by the speed at which they can collect input data and process it.
As an example, the \gls{nist} beacon outputs every 60 seconds~\cite{nistbeacon}.

The \emph{transparent authority} is also limited by their input source, but in their case the limitation is much greater.
They require the source to be published by a third party before they can start their computation, which puts the frequency out of their control.
This is best exemplified with the examples that use the bitcoin blockchain as source~\cite{bonneau2015bitcoin, bentov2016bitcoin, bunz2017proofsof}.
The bitcoin blockchain is designed to produce a new block at a set rate~\cite{nakamoto2008bitcoin}, which aims for a new block every 10 minutes.
This frequency may be too low for some use cases, in which case another type of \emph{entropy sourcing} would be preferable.

By contrast, the frequency of \emph{Specialized MPC} beacons is completely dependent on the participants.
As the randomness is produced from their inputs, the frequency can be as low or high as the participants decide to set it, having a lower bound of the communication latency between participants.

Another performance metric is the initial latency, i.e.\ the time it takes from starting the beacon protocol to the first outcome is produced.
This bootstrapping time is usually insignificant since beacons are intended to be services running for a greater period.
The initial latency is determined by two things:
\begin{eromanate*}
    \item the time it takes to collect input $I_t$, and
    \item the time it takes to perform the computation $f(I_t)$.
\end{eromanate*}
Publishing the outcome is not considered in this metric, since the outcome effectively is available as soon as it is computed.
Since the computation time of $f(I_t)$ affects the latency, beacons using delay functions such as \citealias{bunz2017proofsof} will often have a longer bootstrapping period than beacons with less time-consuming computations such as \citealias{bentov2016bitcoin}.
