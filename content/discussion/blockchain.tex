\subsection{Blockchain Technology in Beacons}

Blockchains are used widely within the beacon literature, both as an source of input~\cite{bonneau2015bitcoin, bentov2016bitcoin, bunz2017proofsof} and as a platform to run a beacon~\cite{randao, bunz2017proofsof}.
The blockchain has a strong foundation of verifiability that makes it easier to run a trustworthy service like a randomness beacon.

The tendency to use the blockchain as input originates with \citet{bonneau2015bitcoin}, who presented the first beacon driven by the bitcoin blockchain.
They showed how you could extract large amounts of entropy from a single block on the blockchain, and how such a beacon could be attacked.
Despite the fact that the blockchain has been shown to be malleable~\cite{pierrot2016malleability}, successful attempts to mitigate this has been developed, e.g.\ by using delay functions to make manipulation negligible due to significantly increased cost~\cite{bunz2017proofsof}.

The input is extracted from the block hashes, that contain a large amount of entropy due to the proof-of-work used in the bitcoin blockchain.
However, this also ties the beacon and its entropy to a consensus protocol that consumes vast amounts of energy~\cite{bitcoinenergy}.

While this does add further utility value to the proof-of-work protocol, it does not change the fact that tying a beacon to such an energy drain may not be sustainable long-term.
While we could consider using blockchains with different consensus protocols, this presents other problems.
Much of the entropy in the blockchain is in the proof-of-work puzzles~\cite{bonneau2015bitcoin}, which would not be available with alternative consensus protocols.
In proof-of-work the entropy is guaranteed to be bound by the difficulty of mining a new block, since there otherwise would be a shortcut to mining, undermining the whole idea of proof-of-work crypto currencies.

One could also theorize that the decentralized nature of the blockchain has had an effect on its use within beacon literature.
The public permissionless setting of blockchains like bitcoin, consits of mutually distrusting participants, and the network solves most of the issues that arise from it.
Within public randomness we typically do not want to rely on any single third party that can act maliciously. This makes a decentralized network a useful source of input, and perhaps even as a platform for executing the beacon protocol.

A downside to using blockchains as sources of input to randomness beacons, is the probabilistic nature of the guarantees a blockchain makes.
At any given time, the state of a blockchain like bitcoin may be rewritten by an adversary, provided they posses the necessary resources.
In the world of financial transactions this is mitigated by requiring a number of blocks to be build on top of the one containing a given transaction, \enquote{confirming} the transaction.
This makes manipulation of the block containing the transaction improbable due the orders of magnitude higher cost, but not impossible.
The same can be said about beacons using the blockchain as input.
If a beacon has outputted some value based on a block, is the output then still valid if the input block is replaced in the blockchain?
In this scenario the outcome would loose its public verifiability, and thereby be untrustworthy.

However, due to the monetary value of crypto currencies like bitcoin, the cost of manipulation of a blockchain used as input for a randomness beacon, will likely be much higher than the gain of manipulating the beacon outcome.
This is especially true when applying delay functions as \citet{bunz2017proofsof} does in \citetalias{bunz2017proofsof}, where they can even adjust the delay, based on the users stake in the outcome of the beacon.

