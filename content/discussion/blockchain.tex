\subsection{Blockchains}
\stefan{title not self-explaining enough}

Blockchains are used widely within the beacon litterature, both as an source of input~\cite{bonneau2015bitcoin, bentov2016bitcoin, bunz2017proofsof} and as a platform to run a beacon~\cite{randao, bunz2017proofsof}.
The blockchain has a strong foundation of verifiability that makes it easier to run a trustworthy service like a randomness beacon.

The tendency to use the blockchain as input originates with \citet{bonneau2015bitcoin}, who presented the first bitcoin beacon.
They showed how you could extract large amounts of entropy from a single block on the blockchain, and how such a beacon could be attacked.
Despite the fact that the blockchain has been shown to be malleable to a certain degree~\cite{pierrot2016malleability}, successful attempts to overcome this fact has been shown~\cite{bunz2017proofsof}.

This entropy is extracted from the block hashes, that contain a large amount of entropy due to the proof-of-work used in the bitcoin blockchain.
However, this also ties the beacon and its entropy to a consensus protocol that consumes vast amounts of energy~\cite{bitcoinenergy}.

While this does add further utility value to the proof-of-work protocol, it does not change the fact that tying a beacon to such an energy drain may not be sustainable long-term.
While we could consider using blockchains with different consensus protocols, this presents other problems.
Much of the entropy in the blockchain is tied to the solutions miners find to the proof-of-work puzzles~\cite{bonneau2015bitcoin}, which would not be available with alternative consensus protocols.
This reduction could also lead to a similar reduction in the use cases of the beacon.

One could also theorize that the decentralized nature of the blockchain has also had an effect on its use within beacon literature.
Since we typically want to avoid relying any single third party that can act maliciously, it makes sense to use a decentralized network as a source of input.

