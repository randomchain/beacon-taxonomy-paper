\section{Introduction}\label{cha:introduction}

Random numbers are useful, important, and even critical, prerequisites in many areas. Lotteries, sport and military drafts, games and gambling all rely on random numbers. Cryptographic schemes used in e.g.\ modern communication contain constants that exhibit randomness. Also in academia, random numbers are required in e.g.\ randomized trials and simulations of unpredictable processes, and in computer science, randomized algorithms are one use case.

Some of these examples happen in a setting where people depend on the outcome of the random draw. In other words, they have a stake in the value of the random number --- this can for example be a monetary prize, an advantage or disadvantage, or in the case of cryptography, the random number helps guarding confidential information. Thus, the quality, or integrity, of such number is important. Essentially, the entity generating the random number should be trustworthy, in the sense that they will not manipulate the outcome to anyone's advantage or disadvantage. However, trustworthiness is difficult to obtain. There are numerous examples of traditional lotteries manipulated --- recently, the information security director of the American Multi-State Lottery Association, Eddie Tipton, was charged with installing a rootkit on the computer generating the random numbers, resulting in him winning the \$14.3 million prize~\cite{bbclotteryexploit}. Another example is from 1980, where the host of lottery television broadcast switched out some of the ping pong balls with slightly heavier ones, resulting in a highly biased draw~\cite{nickperry-lotteryexploit}.

NSA leaks in the past decade further aggravates the question of trust. Massive manipulation and eavesdropping as well as backdoors to cryptographic encryption schemes were revealed. So even if the entity generating the random numbers is trusted, other circumstances may degrade the trust. In any case, having blind faith in an entity is not desireable. Even though they describe their generation process, we have no proof that a random number is generated as they claim it is.

%As an example, the \gls{nist} runs a service that provides randomness. To generate the random value, they use a setup that measures quantum mechanical effects.
%However, there is no proof that the value has been generated by the quantum mechanical procedure as they claim. Users have to blindly trust that the value is generated as described.
%Furthermore, the reputation of the \gls{nist} is somewhat tarnished, as they were involved in the publication of the Dual Elliptic Curve PRNG standard, which they now have admitted contained a back door~\cite{nist2014backdoor}. This makes the value hard to trust for randomness that you do not want the US Government to be able to bias.

Imagine using a random number generator that somehow cryptographically signs its output and provides a verifiable proof saying that it was really generated  as they claim. In such a case, an entity performing a draw could keep requesting a value until a \enquote{satisfactory} value is found. This value is signed and has a proof --- but it is still biased.

The solution could be to use a \emph{randomness beacon}. Michael Rabin introduced this term in 1983~\cite{rabin1983transaction} to describe a trusted public service that provides fresh randomness at regular intervals. An entity wishing to use a random number should then: (i) announce a function mapping the beacon output space to their desired space; this desired space could for example be lottery numbers between 1 and 36, and (ii) announce a specific time in the future at which point the beacon output will be used.

This tactic delegates the trust issue to the randomness beacon. The entity commits to using a future unknown random number which effectively makes the entity unable to keep requesting values until a satisfactory value is found. Furthermore, anyone can see the same random value of the beacon and verify the outcome by using the announced mapping (and of course verifying the mapping is fair).

This still leaves the problem of trusting the beacon to not be biasing the random number. We find it desirable to investigate such concept of trust. While some literature exists on this topic of trustable randomness beacons, we have not been able to find extensive research on it. The existing research is however promising, and this paper will therefore try to classify these to give an overview of the topic.

\mtjnote{TODO: What this paper is about.}