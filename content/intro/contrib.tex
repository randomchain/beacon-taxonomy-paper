\subsection*{Contributions}\label{subsec:contributions}
% comparison parameters
In this paper, we produce a general randomness beacon definition. Afterwards, we create a tangible set of properties which can be used to classify different approaches to creating randomness beacons.
We will be focusing on trust, i.e.\ how a given user of the randomness beacon can trust it, and if they should trust it.
Alongside picking out shortcomings of the different approaches, we will examine where they excel compared to both each other and current solutions for public randomness.

% actual taxonomy
To our knowledge, we provide the first taxonomy of such implementations accompanied by theoretical solutions and ideas, which seek to improve the trustability of randomness sources.

The rise of cryptocurrencies, and more importantly blockchain technologies, has spurred interest in establishing trust in a setting where participants are mutually distrusting.
Randomness beacons display a similar setting, and thus many approaches utilize the blockchain technology.
As such, blockchain technology will be discussed in relation to randomness beacons.

\paragraph{The paper is organized as follows:}
In \cref{sec:use_cases_of_randomness_beacons} we describe the real-world context of randomness beacons by showing motivating use cases.
We specify the properties and settings of randomness beacons in \cref{sec:beacons}, which will be the foundation of classifying current randomness beacon approaches.

In \cref{sec:classification} we define three archetypes of randomness beacons, which loosely describes the prevalent trends in such implementations.
We then present, discuss, and classify existing approaches alongside non-beacon approaches to generating randomness.
Lastly, we conclude upon this work in \cref{sec:conclusion}.

