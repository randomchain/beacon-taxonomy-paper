\section{Introduction}\label{cha:introduction}

% Random numbers are useful, important, and necessary for many things.
The ability to generate random numbers is an important, even critical, prerequisite in many areas. Randomness has been used for quite some time.
Even before computers, randomness was used in ancient Athens, Greece to randomly sample citizens of power in a democracy.
Lotteries have been around for a long time, too, and depend on randomly selecting winners.

% Randomness is in particular important for security
Today, there are many additional uses of randomness.
Cryptographically securing information heavily relies on random numbers, and some new fair consensus protocols are based on randomness.
Since many use cases directly involve money, benefits or disadvantages, and securing secrets, adversaries have an obvious interest in manipulating the outcome of a random number generator to their own advantage.
Therefore, we must make sure that the acclaimed randomness can be trusted.

% TODO: Give some example like lotto (and perhaps another)
\textcolor{orange}{TODO: Given an example or two of a rigged lottery. For example the guy who had access to the computer generating the lotto numbers in some US state, where he installed a rootkit in less than 2 minutes and won the next lotto.}

% TODO: However, creating secure randomness is not easy. Give an example of the NIST beacon insecurity and unavailability.
Creating secure randomness is not easy.
One approach is creating a centralized service providing randomness either on-demand, like random.org, or in intervals, like the NIST beacon.
However, this centralized approach explicitly requires that the service provide can be trusted.
The \gls{nist} beacon is perhaps the most recognized randomness beacon available.
It is run by the US Government, and broadcasts 512 random bits every minute.
However, there is no proof that the bits have been generated by the quantum machine as they claim.
The \gls{nist} beacon has been involved in the publication of the Dual Elliptic Curve PRNG standard, which \gls{nist} now have admitted contained a back door.
Furthermore, the \gls{nist} beacon was taken offline during the 2013 US Government shutdown \mtjnote{Sources are in [onbitcoin] ref.\ 19 and 34}.

% General scare. "Mistrust" is rising.
This example shows that centralized randomness may not be trusted. In the past decade, we have seen numerous leaks revealing aggressive espionage and misuse of information by authorities have surfaced the past decade.
Large corporations and authorities, have on several occasions misused the trust of the people.
In the light of events such as the unveiling of mass surveillance by the \gls{nsa} or their numerous back doors to commonly used systems, a new era of mistrust has begun. If we have not already redefined our picture of an adversary, we certainly should:

\begin{displayquote}[\cite{nakamoto2008bitcoin}][\hfill--- Philip Rogaway]
    \enquote{At this point, I think we would do well to put ourselves in the mindset of a real adversary, not a notional one:
    the well-funded intelligence agency, the profit-obsessed multinational, the drug cartel.
    You have an enormous budget. You control lots of infrastructure.
    You have teams of attorneys more than willing to interpret the law creatively.
    You have a huge portfolio of zerodays. You have a mountain of self-righteous conviction.
    Your aim is to Collect it All, Exploit it All, Know it All.}
\end{displayquote}

As an anti-thesis to centralized organizations and authorities, there has been a surge of interest in decentralization and verifiability. This has roots in the \emph{cypherpunk} culture, which rose in popularity in the early days of the Internet.
These activists advocate strong privacy tools and widespread use of strong cryptography to combat privacy-invading practices of governments and corporations \mtjnote{Source: A Cypherpunk's manifesto: \url{https://www.activism.net/cypherpunk/manifesto.html}}.

\textcolor{orange}{TODO: Since many things rely on strong cryptography and randomness, we need to make sure that the randomness can be trusted. It is obviously an important issue, which has not received much attention in the literature.}
The security of the Internet is based on strong cryptography, and cryptographic protocols are often (TODO: always?) based on randomness.
If the randomness is broken, the security property will not hold. \gls{nsa} has spread much doubt about the security of cryptographic protocols.

%Many new trustless and decentralized technologies have emerged, most prominently the blockchain and consensus protocols which has brought along Bitcoin and other decentralized virtual currencies.

There are several approaches to providing a trustable randomness source; however, to the best of our knowledge, these solutions have not yet been classified, analysed, and compared.
Furthermore, parameters for comparison have not been established to facilitate such a taxonomy.

\subsection*{Contributions}\label{subsec:contributions}
% comparison parameters
In this paper, we establish a tangible set of parameters which can be used to classify different approaches to implement trustable randomness.
% actual taxonomy
Moreover, we provide the first taxonomy of such implementations alongside theoretical solutions and idea, which seeks to improve the trustability of randomness sources.

% blockchain magic to the rescue
We also introduce new ways of improving existing solutions, by harnessing the immutability and consensus power of blockchain networks.

% Today, there exists some services, but only a few, that make random numbers.
%Usually, high-quality randomness is generated by combining a hardware true random number generator (TRNG) and a cryptographically secure pseudo-random number generator (PRNG). The entropy of a TRNG is usually generated based on output from a hardware module utilizing physical properties. %However, it is impossible to prove that a specific output indeed has been generated by this TRNG\@. Even with such a proof, it is impossible to guarantee that the output was not carefully selected from a huge number of outputs of the TRNG\@.

%A few services exist, such as \gls{nist} or Random.org, which

%Indeed, randomness is either generated in a centralized and non-verifiable fashion by an authority such as \gls{nist} \mtjnote[inline]{source}, or by the people who need randomness.
%With these approaches, even though they explicitly describe their protocol of obtaining randomness, we cannot guarantee the protocol was followed as described.
%We have to trust them. \mtjnote[inline]{Rewrite and find source}

% Taken from [milliondollarcurve], section 2.2
