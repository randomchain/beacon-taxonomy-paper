\section{Introduction}\label{cha:introduction}

% Random numbers
The ability to generate random numbers is an important, even critical, prerequisite for many use cases.
Traditionally, lotteries have depended on randomness, and as the usage of cryptography continues to rise, random number generation is a more than ever crucial aspect for security.
Furthermore, new use cases of randomness have emerged, e.g.\ driving fair consensus protocols.

Many uses cases of randomness involve money and security, this dependence merits the question of whether we can trust that the acclaimed \enquote{randomness} is truly random and has not been manipulated to benefit some adversary.

% General scare. "Mistrust" is rising.
Numerous leaks revealing aggressive espionage and misuse of information by authorities have surfaced the past decade.
Large corporations and authorities, have on several occasions misused the trust of the people.
In the light of events such as the unveiling of mass surveillance by the \gls{nsa} or their numerous backdoors to commonly used systems, a new era of mistrust has begun.

Indeed, randomness is either generated in a centralized and non-verifiable fashion by an authority such as \gls{nist} \mtjnote[inline]{source}, or by the people who need randomness.
With these approaches, even though they explicitly describe their protocol of obtaining randomness, we cannot guarantee the protocol was followed as described.
We have to trust them. \mtjnote[inline]{Rewrite and find source}

As an anti-thesis to the centralized organizations and authorities, there has been a surge of interest in decentralization and verifiability.
Many new trustless and decentralized technologies have emerged, most prominently the blockchain which has brought along \gls{btc} and other decentralized virtual currencies.

There are several approaches to providing a trustable randomness source; however, to the best of our knowledge, these solutions have not yet been classified, analysed, and compared.
Furthermore, parameters for comparison have not been established to facilitate such a taxonomy.

\subsection*{Contributions}\label{subsec:contributions}
% comparison parameters
In this paper, we establish a tangible set of parameters which can be used to classify different approaches to implement trustable randomness.
% actual taxonomy
Moreover, we provide the first taxonomy of such implementations alongside theoretical solutions and idea, which seeks to improve the trustability of randomness sources.

% blockchain magic to the rescue
We also introduce new ways of improving existing solutions, by harnessing the immutability and consensus power of blockchain networks.

