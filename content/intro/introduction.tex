\section{Introduction}\label{cha:introduction}

% General scare. "Mistrust" is rising.
Many leaks revealing aggressive espionage and misuse of information by authorities have surfaced the past decade. Now, more than ever, people are suspicious of large corporations and authorities, especially regarding trust. The financial crisis of 2008 and numerous leaks by e.g.\ WikiLeaks have fueled the fire for a debate about the issues of trust and centralization. Numerous leaks have shown that large authorities have wiretapped, broken encryption schemes, and compromised security measures in order to collect information.

As an anti-thesis to these centralized authorities, there has been a surge of interest in decentralization and verifiability. Many new trustless and decentralized technologies have emerged, most prominently perhaps Bitcoin and other decentralized currencies.

% Random numbers
Much today is based on randomness, and the usage of random numbers is crucial in many applications. Much critical infrastructure depends on cryptographic secure protocols based on random numbers; lotteries (todo: define different types of lotteries, e.g.\ \enquote{Lotto}, military lotteries, etc.) depends on random draws; voters for a particular cause in Switzerland are randomly selected; election voting recounts are randomly selected in the US\@; some consensus protocols depend on selecting a random participant; and many more. Yet, trusted random number generation is not widely used for these applications. 
% I hate the examples part of this paragraph.

\subsection{Trusted Random Number Generation} % Maybe a subsection is not needed for this. Maybe it's just a paragraph.
Todo: What is trusted randomness? E.g.\ what is decentralized verifiable
% What is trusted randomness?
  % e.g. decentralized verifiable


  %, especially regarding whether they will skew results. The National Security Agency (NSA) has been the center of many leaks detailing their practices. Their capabilities to eavesdrop world wide have been brought to light, which revealed tactics to wiretap, break encryption schemes, and compromise security measures.

  %The world of today is increasingly aware of trust. Now, more than ever, many people are suspicious of who to trust. Numerous leaks have revealed aggressive espionage by governments, and as media coverage is dedicated to the aftermath of numerous leaks by WikiLeaks, we grow more suspicious of sharing our private information. We don't trust authorities as we now know they may be corrupt.

  %Decentralized approaches like Bitcoin rises in popularity. People don't trust large organizations.

% Why do we need publicly verifiable randomness
%Some of these issues are dependent on randomness. 
%A subclass of these issues are related to trusting authorities to make fair decisions that will not necessarily benefit the authority in question. Examples of this are generating parameters used in cryptographic encryption schemes and 

% Why can we not trust a centralized randomness generator?

% There are many ways to mitigate these problems. Therefore we will create a taxonomy / survey of the most prominent papers.

% "To our knowledge, no survey exists"
