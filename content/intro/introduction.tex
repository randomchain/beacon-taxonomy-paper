\section{Introduction}\label{cha:introduction}

% Why do we need randomness?
Random numbers are used for many purposes in todays world. Lotteries rely on random draws, cryptographic schemes.



% General scare
Many leaks revealing aggressive espionage by authorities have surfaced the past decade. Now, more than ever, people are suspicious of large corporations and authorities, especially regarding whether they will misuse information for their own gain.
The National Security Agency (NSA) has been the center of many leaks detailing their practices. Their capabilities to eavesdrop world wide have been brought to light, which revealed tactics to wiretap, break encryption schemes, and compromise security meassures.

After numerous leaks revealing aggressive tactics to wiretap, break encryption schemes, and compromise security, there has been much interest in issues of trust and centralization.

%The world of today is increasingly aware of trust. Now, more than ever, many people are suspicious of who to trust. Numerous leaks have revealed aggressive espionage by governments, and as media coverage is dedicated to the aftermath of numerous leaks by WikiLeaks, we grow more suspicious of sharing our private information. We don't trust authorities as we now know they may be corrupt.

%Decentralized approaches like Bitcoin rises in popularity. People don't trust large organizations.

% Why do we need publicly verifiable randomness
Some of these issues are dependent on randomness. 
A subclass of these issues are related to trusting authorities to make fair decisions that will not necessarily benefit the authority in question. Examples of this are generating parameters used in cryptographic encryption schemes and 

% Why can we not trust a centralized randomness generator?

% There are many ways to mitigate these problems. Therefore we will create a taxonomy / survey of the most prominent papers.

% "To our knowledge, no survey exists"
