\section{Introduction}\label{cha:introduction}

% Random numbers are useful, important, and necessary for many things.
Random numbers are useful, important, and even critical, prerequisites in many areas.
%Even before computers, randomness was used in ancient Athens, Greece to randomly sample citizens of power in a democracy.
%Lotteries have been around for a long time, too, and depend on randomly selecting winners, for example by picking numbered balls from a rotating cage.
% Randomness is in particular important for security
%Today, there are many additional uses of randomness.
%Cryptographically securing and sharing information heavily relies on random numbers, and some new fair consensus protocols are based on randomness.
There are a number of interesting use cases that require or could benefit from randomness that is publicly available. For example, the numbered balls of lotteries are often drawn on live tv, which is a way of publicizing the random draw process and more importantly prove that the process has not been manipulated.

The \gls{nba} annually holds a lottery, which decides the order of which teams get to select the top amateur players who are ready to turn into professionals. Even though the draw is performed by picking envelopes shuffled in a transparent spinning drum on live tv, fans have claimed for many years that the lottery is rigged to favor a single team~\cite{princeton2016}. Famous conspiracy theories revolve around bending corners on the envelopers, or keeping a single envelope in the freezer. It is hard to dispel these theories, and even harder to prove that the lottery is fair.

In 1969, this type of lottery was performed in the US\@. However, it was slightly more serious --- it was a conscription lottery for the US Military, and many would be sent to fight in the Vietnam war. Similarly, small capsules labelled with each day of the year were picked from a plastic drum on live tv. The ordering of the dates drawn and the birthdays of all men then determined the conscription order. The lottery was quickly proven to be botched; statisticians noticed an anomalous pattern. The broadcast was reviewed, and the drum was always turned an even number of times, such that capsules on top stayed on top. In other words, there was not sufficient mixing to make it a statistically random draw~\cite{princeton2016}.

% TODO: Give some example like lotto (and perhaps another)
\textcolor{orange}{TODO: The guy who had access to the computer generating the lotto numbers in some US state, where he installed a rootkit in less than 2 minutes and won the next lotto.}

These examples show that it is difficult to generate public randomness and convince the public that it is truly random --- the process might not be sufficiently random, or the public might not believe it.

% TODO: However, creating secure randomness is not easy. Give an example of the NIST beacon insecurity and unavailability.
Even today with computers available, creating public randomness is not easy. Rabin introduced the term \emph{beacon}~\cite{rabin1983transaction}, a trusted service to provide fresh randomness at regular intervals. The idea is that nobody can predict the next value in advance.

The \gls{nist} beacon is perhaps the most recognized randomness beacon available.
It is run by the US Government, and broadcasts 512 random bits every minute.
If this beacon could be trusted, it would be universally usable. The random bits are trivially converted into numbers, and are easily usable by national lotteries and even in your local bingo club every Wednesday night.
However, there is no proof that the bits have been generated by the quantum machine as they claim. You have to trust that somewhere in a building in Maryland, the bits are generated as described.
Their reputation is somewhat tarnished, as they were involved in the publication of the Dual Elliptic Curve PRNG standard, which they now have admitted contained a back door~\cite{nist2014backdoor}.
Furthermore, a centralized beacon may be unreliable --- the \gls{nist} beacon was offline during the 2013 US Government shutdown~\cite{bonneau2015bitcoin}.

\textcolor{orange}{TODO: Randomness is also important in cryptography.}

% Adversaries.
As seen, use cases of randomness often involve either money, benefits or disadvantages, and securing secrets. Therefore, adversaries have an obvious interest in manipulating the outcome of a random number generator to their own advantage.
This merits the question of whether the acclaimed randomness can be \emph{trusted}.

% General scare. "Mistrust" is rising.
In the past decade, we have seen numerous leaks revealing aggressive espionage and misuse of information by authorities have surfaced the past decade.
Large corporations and authorities, have on several occasions misused the trust of the people.
In the light of events such as the unveiling of mass surveillance by the \gls{nsa} or their numerous back doors to commonly used systems, a new era of mistrust has begun. If we have not already redefined our picture of an adversary, we certainly should. It is no longer necessarily a single hacker, but instead the well-funded intelligence agency or the profit-obsessed multinational company. They have enormous budgets, control vast amounts of infrastructure, and have attorneys more than willing to interpret the law creatively. They have a huge portfolio of zerodays. Their aim is to Collect it All, Exploit it All, Know it All~\cite{rogaway2015moral}.

%As an anti-thesis to centralized organizations and authorities, there has been a surge of interest in decentralization and verifiability. This has roots in the \emph{cypherpunk} culture, which rose in popularity in the early days of the Internet.
%These activists advocate strong privacy tools and widespread use of strong cryptography to combat privacy-invading practices of governments and corporations~\cite{hughes1993cypherpunk}.

Since many things rely on strong cryptography and randomness, we need to make sure that the randomness can be trusted. If the randomness is broken, adversaries can get an unfair advantage --- \gls{nsa} has spread much doubt about the security of some cryptographic protocols. It is an important issue, which has received some, but not much attention in the literature~\cite{lenstra2015random}.

\textcolor{orange}{TODO: What is the problem? Do we want a beacon with no trusted parties? Do we also want it decentralized? because then we might get problems explaining which papers we have included.}

There are several approaches to providing a trustable randomness source; however, to the best of our knowledge, these solutions have not yet been classified, analysed, and compared.
Furthermore, parameters for comparison have not been established to facilitate such a taxonomy.

\subsection*{Contributions}\label{subsec:contributions}
% comparison parameters
In this paper, we establish a tangible set of parameters which can be used to classify different approaches to implement trustable randomness.
% actual taxonomy
Moreover, we provide the first taxonomy of such implementations alongside theoretical solutions and idea, which seeks to improve the trustability of randomness sources.

% blockchain magic to the rescue
We also introduce new ways of improving existing solutions, by harnessing the immutability and consensus power of blockchain networks.

% Today, there exists some services, but only a few, that make random numbers.
%Usually, high-quality randomness is generated by combining a hardware true random number generator (TRNG) and a cryptographically secure pseudo-random number generator (PRNG). The entropy of a TRNG is usually generated based on output from a hardware module utilizing physical properties. %However, it is impossible to prove that a specific output indeed has been generated by this TRNG\@. Even with such a proof, it is impossible to guarantee that the output was not carefully selected from a huge number of outputs of the TRNG\@.

%A few services exist, such as \gls{nist} or Random.org, which

%Indeed, randomness is either generated in a centralized and non-verifiable fashion by an authority such as \gls{nist} \mtjnote[inline]{source}, or by the people who need randomness.
%With these approaches, even though they explicitly describe their protocol of obtaining randomness, we cannot guarantee the protocol was followed as described.
%We have to trust them. \mtjnote[inline]{Rewrite and find source}

% Taken from [milliondollarcurve], section 2.2



% Examples:
% NBA annual draft lottery. See OnBitcoin
% US conscription lottery, vietnam war. OnBitcoin
